\documentclass[letterpaper, 12pt]{article}
\usepackage[english]{babel}
\usepackage[letterpaper,top=2cm,bottom=2cm,left=3cm,right=3cm,marginparwidth=1.75cm]{geometry}
\usepackage[colorlinks=true, allcolors=blue]{hyperref}
\usepackage{graphicx}
\graphicspath{{Figures/}{./}}
\usepackage{amsmath}
\usepackage{amssymb}
\usepackage{amsthm}
\usepackage{indentfirst}
\setlength{\parindent}{20pt}
\usepackage{siunitx}
\usepackage[justification=centering]{caption}
\usepackage{float}
\usepackage{longtable}
\usepackage{tabularray}

\nocite{}

\title{An Investigation of the Moments\\ About a Horizontal Rigid Beam \\ IB Physics SL}
\author{Ethan Chen}

\begin{document}

\maketitle

\begin{center}
    Mr. Shaw
\end{center}

\section{Experimental plan}

\subsection{Manipulated variable}
The manipulated variable is the distance of the weights from the pivot point /\unit{m}.
In this lab, the manipulated variable will be changed by moving the position of the weights to the right.
This change will be measured by the ruler that the weights are hanging from.

\subsection{Responding variable}
The responding variable is the amount of force that the spring balance measures /\unit{N}.
The responding variable will be measured using a spring balance.


\subsection{Controlled variables}

\begin{table}[H]
    \centering
    \begin{tabular}{|l|l|l|}
        \hline
        Condition                                                                                                  & How to control it                                                                                                                                                               & Why it must be controlled                                                                                                                                                                                                                                                           \\
        \hline
        \begin{tabular}[c]{@{}l@{}}The position\\of the pivot\\and spring \\balance\\along the ruler.\end{tabular} & \begin{tabular}[c]{@{}l@{}}Recording each\\trial with the pivot\\and the spring\\balance consistently\\positioned at the exact\\same location along\\the ruler.\end{tabular}    & \begin{tabular}[c]{@{}l@{}}Varying the position\\of these objects will\\change the distances\\between the pivot to\\the ruler's center of\\mass, the weights,\\and the spring balance.\\This will ultimately\\cause inconsistencies\\in the calculated torque\\values.\end{tabular} \\
        \hline
        \begin{tabular}[c]{@{}l@{}}The number\\of weights\\suspended on\\the ruler.\end{tabular}                   & \begin{tabular}[c]{@{}l@{}}Recording each trial\\with the same number\\of weights\\suspended from\\the ruler (in this\\experiment the weight\\will be 10\unit{N}).\end{tabular} & \begin{tabular}[c]{@{}l@{}}Varying the number\\of weights suspended\\on the ruler will cause\\the gravitational force\\of the weights\\to vary, affecting\\both calculations\\for net force and\\calculations\\for net torque.\end{tabular}                                         \\
        \hline
    \end{tabular}
    \caption{Controlled variables in the experiment}
    \label{table:controlled_variables}
\end{table}


\section{Background Information}

\subsection{Horizontal ruler requirement}

If this experiment is to be physically conducted, then there is a need for the
ruler or beam to be horizontal. This is because the torques acting on the beam only come from the the force
component perpendicular to the beam, therefore using a horizontal beam would
then simplify the analysis, as the calculation of the components of
the forces perpendicular to the beam would not be required.

What is meant by the beam/ruler to be horizontal is visually shown by the ruler in
Figure \ref*{fig:horizontal}.

\begin{figure}[H]
    \centering
    \includegraphics[width=0.65\textwidth]{horizontal}
    \caption{Horizontal beam/ruler in the apparatus of the investigation.}
    \label{fig:horizontal}
\end{figure}

\subsection{Conditions for static equilibrium and derivations of required equations}

The following conditions need to be met for static equilibrium:
\begin{itemize}
    \item The sum of all forces acting on the system must be equal to zero ($\sum F = 0$).
    \item The sum of all torques with respect to some point in the system must be equal to zero ($\sum \tau = 0$).
\end{itemize}


\begin{figure}[H]
    \centering
    \includegraphics[width=0.65\textwidth]{labelled apparatus and variables}
    \caption{Apparatus labelled with variables along with descriptions of forces acting in the system}
    \label{fig:labapparatus}
\end{figure}

With reference to the variables in Figure \ref*{fig:labapparatus}, we can derive the following equations.

\subsubsection*{Forces acting on the beam}
\begin{align*}
     & \sum F                      = 0
    \\
     & \vec{R} + \vec{L} + \vec{F} = 0
\end{align*}

\subsubsection*{Torques acting on the beam relative to support pivot (orange triangle in Figure \ref*{fig:labapparatus})}

\begin{align*}
     & \text{Assigning clockwise spin as the positive direction}
    \\
     & \sum \tau = 0
    \\
     & \left(\left|\vec{W}\right|\cdot D\right) + \left(\left|\vec{L}\right|\cdot D_1\right) - \left(\left|\vec{F}\right|\cdot D_2\right) = 0
\end{align*}

\section{Evidence}

\subsection{Determination of uncertainty in force measured by spring balance}

Setting the weights the 0.801\unit{m} along the ruler, as shown in Figure
\ref*{fig:uncRef}, the range of the force measured by the spring balance is
0.5\unit{N}, with a minimum value of 9.2\unit{N} and a maximum value of
9.7\unit{N}.

\begin{figure}[H]
    \centering
    \includegraphics[width=0.65\textwidth]{uncRef}
    \caption{Positioning of weight in apparatus for uncertainty calculation of spring balance}
    \label{fig:uncRef}
\end{figure}

Therefore, by dividing this range by 2, the uncertainty of the force probe is
determined to be 0.3\unit{N}.

\subsection{Collected raw data}


\begin{longtable}{|c|c|}
    \caption{Table of Position of weights along ruler with respective values of Force measured by spring scale, given that the weight suspended is 10\unit{N}, pivot position at 0.10m, and spring balance position at 0.90m}                                                                                                                                            \\
    \hline
    \multicolumn{1}{|l|}{\begin{tabular}[c]{@{}l@{}}Position of weights\\along ruler /\unit{m}\\($\Delta \unit{m} \pm 0.001\unit{m}$)\end{tabular}} & \multicolumn{1}{l|}{\begin{tabular}[c]{@{}l@{}}Force measured by\\spring scale /\unit{N}\\($\Delta \unit{m} \pm 0.5\unit{N}$)\end{tabular}}  \endfirsthead
    \hline
    $0.200$                                                                                                                                                                     & $2.0$                                                                                                                                                                                  \\
    \hline
    $0.210$                                                                                                                                                                     & $2.1$                                                                                                                                                                                  \\
    \hline
    $0.220$                                                                                                                                                                     & $2.3$                                                                                                                                                                                  \\
    \hline
    $0.230$                                                                                                                                                                     & $2.4$                                                                                                                                                                                  \\
    \hline
    $0.240$                                                                                                                                                                     & $2.5$                                                                                                                                                                                  \\
    \hline
    $0.250$                                                                                                                                                                     & $2.6$                                                                                                                                                                                  \\
    \hline
    $0.260$                                                                                                                                                                     & $2.8$                                                                                                                                                                                  \\
    \hline
    $0.270$                                                                                                                                                                     & $2.9$                                                                                                                                                                                  \\
    \hline
    $0.280$                                                                                                                                                                     & $3.0$                                                                                                                                                                                  \\
    \hline
    $0.290$                                                                                                                                                                     & $3.1$                                                                                                                                                                                  \\
    \hline
    $0.300$                                                                                                                                                                     & $3.2$                                                                                                                                                                                  \\
    \hline
    $0.310$                                                                                                                                                                     & $3.4$                                                                                                                                                                                  \\
    \hline
    $0.320$                                                                                                                                                                     & $3.5$                                                                                                                                                                                  \\
    \hline
    $0.330$                                                                                                                                                                     & $3.6$                                                                                                                                                                                  \\
    \hline
    $0.340$                                                                                                                                                                     & $3.8$                                                                                                                                                                                  \\
    \hline
    $0.350$                                                                                                                                                                     & $3.9$                                                                                                                                                                                  \\
    \hline
    $0.360$                                                                                                                                                                     & $4.0$                                                                                                                                                                                  \\
    \hline
    $0.370$                                                                                                                                                                     & $4.2$                                                                                                                                                                                  \\
    \hline
    $0.380$                                                                                                                                                                     & $4.3$                                                                                                                                                                                  \\
    \hline
    $0.390$                                                                                                                                                                     & $4.4$                                                                                                                                                                                  \\
    \hline
    $0.400$                                                                                                                                                                     & $4.5$                                                                                                                                                                                  \\
    \hline
    $0.410$                                                                                                                                                                     & $4.6$                                                                                                                                                                                  \\
    \hline
    $0.420$                                                                                                                                                                     & $4.8$                                                                                                                                                                                  \\
    \hline
    $0.430$                                                                                                                                                                     & $4.9$                                                                                                                                                                                  \\
    \hline
    $0.440$                                                                                                                                                                     & $5.0$                                                                                                                                                                                  \\
    \hline
    $0.450$                                                                                                                                                                     & $5.2$                                                                                                                                                                                  \\
    \hline
    $0.460$                                                                                                                                                                     & $5.2$                                                                                                                                                                                  \\
    \hline
    $0.470$                                                                                                                                                                     & $5.4$                                                                                                                                                                                  \\
    \hline
    $0.480$                                                                                                                                                                     & $5.5$                                                                                                                                                                                  \\
    \hline
    $0.490$                                                                                                                                                                     & $5.6$                                                                                                                                                                                  \\
    \hline
    $0.500$                                                                                                                                                                     & $5.8$                                                                                                                                                                                  \\
    \hline
    $0.510$                                                                                                                                                                     & $5.9$                                                                                                                                                                                  \\
    \hline
    $0.520$                                                                                                                                                                     & $6.0$                                                                                                                                                                                  \\
    \hline
    $0.530$                                                                                                                                                                     & $6.2$                                                                                                                                                                                  \\
    \hline
    $0.540$                                                                                                                                                                     & $6.3$                                                                                                                                                                                  \\
    \hline
    $0.550$                                                                                                                                                                     & $6.5$                                                                                                                                                                                  \\
    \hline
    $0.560$                                                                                                                                                                     & $6.7$                                                                                                                                                                                  \\
    \hline
    $0.570$                                                                                                                                                                     & $6.6$                                                                                                                                                                                  \\
    \hline
    $0.580$                                                                                                                                                                     & $6.8$                                                                                                                                                                                  \\
    \hline
    $0.590$                                                                                                                                                                     & $6.8$                                                                                                                                                                                  \\
    \hline
    $0.600$                                                                                                                                                                     & $7.1$                                                                                                                                                                                  \\
    \hline
    $0.610$                                                                                                                                                                     & $7.2$                                                                                                                                                                                  \\
    \hline
    $0.620$                                                                                                                                                                     & $7.3$                                                                                                                                                                                  \\
    \hline
    $0.630$                                                                                                                                                                     & $7.4$                                                                                                                                                                                  \\
    \hline
    $0.640$                                                                                                                                                                     & $7.6$                                                                                                                                                                                  \\
    \hline
    $0.650$                                                                                                                                                                     & $7.7$                                                                                                                                                                                  \\
    \hline
    $0.660$                                                                                                                                                                     & $7.8$                                                                                                                                                                                  \\
    \hline
    $0.670$                                                                                                                                                                     & $7.9$                                                                                                                                                                                  \\
    \hline
    $0.680$                                                                                                                                                                     & $8.0$                                                                                                                                                                                  \\
    \hline
    $0.690$                                                                                                                                                                     & $8.2$                                                                                                                                                                                  \\
    \hline
    $0.700$                                                                                                                                                                     & $8.3$                                                                                                                                                                                  \\
    \hline
    $0.710$                                                                                                                                                                     & $8.4$                                                                                                                                                                                  \\
    \hline
    $0.720$                                                                                                                                                                     & $8.6$                                                                                                                                                                                  \\
    \hline
    $0.730$                                                                                                                                                                     & $8.7$                                                                                                                                                                                  \\
    \hline
    $0.740$                                                                                                                                                                     & $8.8$                                                                                                                                                                                  \\
    \hline
    $0.750$                                                                                                                                                                     & $8.9$                                                                                                                                                                                  \\
    \hline
    $0.760$                                                                                                                                                                     & $9.1$                                                                                                                                                                                  \\
    \hline
    $0.770$                                                                                                                                                                     & $9.2$                                                                                                                                                                                  \\
    \hline
    $0.780$                                                                                                                                                                     & $9.3$                                                                                                                                                                                  \\
    \hline
    $0.790$                                                                                                                                                                     & $9.5$                                                                                                                                                                                  \\
    \hline
    $0.800$                                                                                                                                                                     & $9.6$                                                                                                                                                                                  \\
    \hline
    $0.810$                                                                                                                                                                     & $9.7$                                                                                                                                                                                  \\
    \hline
    $0.820$                                                                                                                                                                     & $9.8$                                                                                                                                                                                  \\
    \hline
    $0.830$                                                                                                                                                                     & $10.0$                                                                                                                                                                                 \\
    \hline
\end{longtable}


\section{Analysis}

\subsection{Processed predicted data}

Taking the trial with the weight positioned at $0.200\unit{m}$ along the ruler,
we can determine the predicted force of the spring balance for this trial.

Because the force applied by the pivot point is unknown but the weight of the ruler is known (1.5\unit{N}),
then only the torque equation defined earlier (shown below) can be used to solve for
the predicted force of the spring balance.
$$
    \left(\left|\vec{W}\right|\cdot D\right) + \left(\left|\vec{L}\right|\cdot D_1\right) - \left(\left|\vec{F}\right|\cdot D_2\right) = 0
$$

\begin{figure}[H]
    \centering
    \includegraphics[width=0.65\textwidth]{distanceref}
    \caption{Reference for distance and positions of objects along ruler}
    \label{fig:distRef}
\end{figure}

The calculation for the predicted force of the spring balance given that
the weight is positioned at $0.200\unit{m}$ along the ruler is shown below,
with distance values originating from Figure \ref*{fig:distRef}

\begin{align*}
     & \left(\left|\vec{W}\right|\cdot D\right) + \left(\left|\vec{L}\right|\cdot D_1\right) - \left(\left|\vec{F}\right|\cdot D_2\right) = 0
\end{align*}

\begin{align*}
    \text{where } & \left|\vec{W}\right| = \text{ the magnitude of the weight of the ruler } /\unit{N}
    \\
                  & \left|\vec{L}\right| = \text{ the magnitude of the weight of the weights suspended on the ruler } /\unit{N}
    \\
                  & \left|\vec{F}\right| = \text{ the magnitude of the force applied onto the ruler by the}
    \\ &\text{ spring balance } /\unit{N}
    \\
                  & D = \text{ the distance between the pivot and the ruler's centre of mass } /\unit{m}
    \\ & D_1 = \text{ the distance between the pivot point and the weights } /\unit{m}
    \\ & D_2 = \text{ the distance between the pivot point and the spring balance } /\unit{m}
\end{align*}

\begin{align*}
    \\
                       & \left(\left|\vec{F}\right|\cdot D_2\right) = \left(\left|\vec{W}\right|\cdot D\right) + \left(\left|\vec{L}\right|\cdot D_1\right)
    \\
    \text{Given that } & \left|\vec{W}\right| = 1.5\unit{N}
    \\
                       & \left|\vec{L}\right| = 10\unit{N}
    \\
                       & D_2 = 0.900\unit{m} - 0.100\unit{m} = 0.800\unit{m}
    \\
                       & D = 0.500\unit{m} - 0.100\unit{m} = 0.400\unit{m}
    \\
                       & D_1 = 0.200\unit{m} - 0.100\unit{m} = 0.100\unit{m}
    \\
                       & (0.800\unit{m})\cdot \left|\vec{F}\right| = (0.400\unit{m})(1.5\unit{N}) + (0.100\unit{m})(10\unit{N})
    \\
                       & \left|\vec{F}\right| = 2.0\unit{N}
\end{align*}

Continuing with these calculations, the graph in Figure \ref*{fig:predGraph} shows the
trend of the predicted force of the spring balance with respect to the distance
between the weights and the pivot point.

\begin{figure}[H]
    \centering
    \includegraphics[width=\textwidth]{predictedGraph.pdf}
    \caption{Predicted Force measured by spring balance /\unit{N} versus Distance between weights and pivot point along ruler /\unit{m}}
    \label{fig:predGraph}
\end{figure}

The trendline of the graph in Figure \ref*{fig:predGraph}
has the equation $y=12.5x-0.5$.


\subsection{Processed experimental data}

Taking the recorded data of the actual force measured by the spring balance
for each trial, we can form a graph and determine the line of best-fit.
This graph is shown in Figure \ref*{fig:expGraph}.

\begin{figure}[H]
    \centering
    \includegraphics[width=\textwidth]{expGraph.pdf}
    \caption{Experimental Force measured by spring balance /\unit{N} versus Distance between weights and pivot point along ruler /\unit{m}}
    \label{fig:expGraph}
\end{figure}

Drawing from the graph, we can determine that generally,
the equation of the best-fit line would be $y=12.635x - 0.5321$.
There will also be:
\begin{itemize}
    \item a trendline of maximum slope ($y = 13.694x - 1.0525$)
    \item a trendline of minimum slope ($y = 11.709x - 0.0301$)
\end{itemize}

\subsection{y-intercept of the graph}

The significance of the y-intercept is that it indicates what force would
be measured by the spring balance if the distance between the weights and
the pivot point was 0. In other words, what the measured force would be when
the weights are suspended right above the pivot point.

\subsubsection{Value based off of predicted data}

Based off the line of best-fit in the graph of Figure \ref*{fig:predGraph}
($y = 12.5x - 0.5$), the y-intercept of the graph is $-0.5\unit{N}$.

What this suggests is that if the weights were suspended at the same
position as the pivot point, then the spring balance would be pushing
\textbf{down} with 0.5\unit{N} of force.

\subsubsection{Value based off of experimental data}

While the line of best-fit yields a y-intercept of -0.5321\unit{N},
this does not equal to the average between the y-intercepts of the
trendlines with maximum and minimum slope that is -0.5413\unit{N}. Therefore,
the latter value will be used instead to deduce that the y-intercept of
the experimental data graph is $-0.5\unit{N} \pm 0.5\unit{N}$.

The calculations for these values are shown below:
\begin{align*}
                  & \text{Average calculation:}
    \\
                  & \overline{b} = \frac{\sum b_i}{n}
    \\
    \text{where } & \overline{b} = \text{ average y-intercept value } /\unit{N}
    \\ & b_i = \text{ y-intercept value from maximum or minimum trendline } /\unit{N}
    \\ & n = \text{ number of values averaged }
    \\
                  & \overline{b} = \frac{-1.0525\unit{N} - 0.0301\unit{N}}{2}
    \\
                  & \overline{b} = -0.5413\unit{N}
    \\
                  & \text{Final y-intercept of experimental data graph calculation:}
    \\
                  & b = \overline{b} \pm \left(\frac{b_{min} - b_{max}}{2}\right)
    \\
    \text{where } & b_{min} = \text{ y-intercept value of trendline with minimum slope } /\unit{N}
    \\ & b_{max} = \text{ y-intercept value of trendline with maximum slope } /\unit{N}
    \\
                  & b = -0.5413\unit{N} \pm \left(\frac{-0.0301\unit{N} + 1.0525\unit{N}}{2}\right)
    \\
                  & b = -0.5\unit{N} \pm 0.5\unit{N}
\end{align*}

What this suggests is that if the weights were suspended at the same
position as the pivot point, then the spring balance would be pushing
\textbf{down} with $-0.5\unit{N} \pm 0.5\unit{N}$ of force.

\subsubsection{Comparison of predicted y-intercept and experimental y-intercept}

Given the experimental value of $-0.5\unit{N} \pm 0.5\unit{N}$,
the predicted value (-0.5\unit{N}) falls within the range of the
experimental value's uncertainty.

However, when doing a mathematical calculation to compare the experimental
value to the predicted value, we will use the experimental
y-intercept which came from the original line of best-fit
(-0.5321\unit{N}).

Below shows the calculation of the percent error between the experimental
y-intercept and predicted y-intercept:
\begin{align*}
     & \%_{err} = \left|\frac{experimental - actual}{actual}\right| \times 100\%
    \\
     & = \left|\frac{-0.5321\unit{N} - (-0.5\unit{N})}{-0.5\unit{N}}\right| \times 100\%
    \\
     & \%_{err} = 6.42\%
    \\
     & \%_{err} = 6\%
\end{align*}

\subsection{Slope of the graph}

The significance of the slope of the graph is that it indicates the rate
at which the force increases when the position of the weights are shifted
towards the right.

\subsubsection{Value based off of predicted data}

Based off the line of best-fit in the graph of Figure \ref*{fig:predGraph}
($y=12.5x-0.5$),
the slope of the graph is $12.5\unit{Nm^{-1}}$.

What this suggests is that for every meter that the suspended weights
are moved to the right along the ruler, the force measured by the
spring balance will increase by $12.5\unit{N}$.

\subsubsection{Value based off of experimental data}

While the line of best-fit yields a slope of $12.635\unit{Nm^{-1}}$,
this does not equal to the average between the slopes of the trendlines
with maximum and minimum slope that is $12.702\unit{Nm^{-1}}$. Therefore,
the latter value will be used instead to deduce that the slope of the
experimental data graph is $13\unit{Nm^{-1}} \pm 1\unit{Nm^{-1}}$.

The calculations for these values are shown below:

\begin{align*}
                  & \text{Average calculation:}
    \\
                  & \overline{m} = \frac{\sum m_i}{n}
    \\
    \text{where } & \overline{m} = \text{ average slope value } /\unit{Nm^{-1}}
    \\ & m_i = \text{ slope value of trendline with maximum or slope } /\unit{Nm^{-1}}
    \\ & n = \text{ number of values averaged }
    \\
                  & \overline{m} = \frac{13.694\unit{Nm^{-1}} + 11.709\unit{Nm^{-1}}}{2}
    \\
                  & \overline{m} = 12.702\unit{Nm^{-1}}
    \\
                  & \text{Final slope of experimental data graph calculation:}
    \\
                  & m = \overline{m} \pm \left(\frac{m_{min} - m_{max}}{2}\right)
    \\
    \text{where } & m_{min} = \text{ slope value of trendline with minimum slope } /\unit{Nm^{-1}}
    \\ & m_{max} = \text{ slope value of trendline with maximum slope } /\unit{Nm^{-1}}
    \\
                  & m = 12.702\unit{Nm^{-1}} \pm \left(\frac{13.694\unit{Nm^{-1}} - 11.709\unit{Nm^{-1}}}{2}\right)
    \\
                  & m = 13\unit{Nm^{-1}} \pm 1\unit{Nm^{-1}}
\end{align*}

What this suggests is that experimentally, the force measured by the
spring balance would increase by $13\unit{N} \pm 1\unit{N}$ for every
meter that the suspended weights are moved to the right.

\subsubsection{Comparison of predicted slope and experimental slope}

Given the experimental value of $13\unit{Nm^{-1}} \pm 1\unit{Nm^{-1}}$,
the predicted value (12.5\unit{Nm^{-1}}) falls within the range of the
experimental value's uncertainty.

However, when doing a mathematical calculation to compare the experimental
value to the predicted value, we will use the experimental slope which
came from the original line of best-fit (12.635\unit{Nm^{-1}}).

Below shows the calculation of the percent error between the experimental
slope and predicted slope:

\begin{align*}
     & \%_{err} = \left|\frac{experimental - actual}{actual}\right| \times 100\%
    \\
     & = \left|\frac{12.635\unit{Nm^{-1}} - 12.5\unit{Nm^{-1}}}{12.5\unit{Nm^{-1}}}\right| \times 100\%
    \\
     & \%_{err} = 1.08\%
    \\
     & \%_{err} = 1\%
\end{align*}


\end{document}